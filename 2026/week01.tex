\subsection{Week 1 (Jan 1--Jan 7)}

\dayentry{2026-01-01}{Divisibility}
\begin{theorem} 
	If $a | c$ and $b | c$ and $\gcd(a, b) = d$, then $ab | cd$. $a, b, c \in \mathbb{Z}$
\end{theorem}
\begin{proof}
	As $d = \gcd(a, b)$, we have that $d = au + bv$ for some $u, v \in \mathbb{Z}$. We now have that $cd = c(au + bv) = cau + cbv$. Since $a | c$ and $b | c$, $c = ax = by$ for some $x, y \in \mathbb{Z}$. Substituing $c$ results in $cd = cau +cbv = (by)au + (ax)bv = (ab)(yu + xv)$. As $u, v, x, y \in \mathbb{Z}$, $yu +xv \in \mathbb{Z}$. By definition of divisibility, $ab | cd$.  
\end{proof}

\dayentry{2026-01-02}{Divisibility}
\begin{theorem}
	If $a|bc$ and $\gcd(a, b) = 1$, then $a|c$. 
\end{theorem}
\begin{proof}
	Assume that $a|bc$ and $\gcd(a, b) = 1$. As $a$ and $b$ are relatively prime, $au + bv = 1$ for some $u, v \in \mathbb{Z}$. If we multiply $au + bv = 1$ by $c$, then we have $cau + cbv = c$. As $a|bc$, $bc = ax$ for some $x \in \mathbb{Z}$. Now, we have that 
	\[
		c = cau + cbv = cau + bcv = cau + (ax)v = a(cu + xv)
	\]
	Since $c, u, x, v \in \mathbb{Z}$ and by the definition of divisibility, $a|c$. 
\end{proof}

\dayentry{2026-01-03}{Divisibility}
\begin{theorem}
	If $\gcd(a, c) = 1$ and $\gcd(b, c) = 1$, then $\gcd(ab, c) = 1$. 
\end {theorem} 	
\begin{proof}
	By way of contradiction, assume that $\gcd(ab, c) \neq 1$. Thus, there exists a prime $p$ such that 
	\[
		p | ab \text{ and } p|c.
	\]
	Since $p$ is prime, $p | a$ or $p | b$. If $p | a$, then $\gcd{a, c} \geq p > 1$, which contradiscts the assumption that $\gcd(a, c) = 1$. If $p | b$, then $\gcd{b, c} \geq p > 1$, which contradicts the assumption that $\gcd(b, c) = 1$. Therefore $ab$ and $c$ share no such prime factors $p$, so $\gcd(ab, c) = 1$. 
\end{proof}

\dayentry{2026-01-04}{Rings}
\begin{theorem}
	If $a + b = a + c$ in a ring $R$, then $b = c$. 
\end{theorem}
\begin{proof}
	Assume that for $a, b, c \in R$, $a + b = a + c$. If we add $-a$ and by using the associtivity property of rings and negatives, we can show that 
	\begin{align*}
		-a + (a + b) = -a + (a + c) \\
		(-a + a) + b = (-a + a) + c \\
		0_R + b = 0_R + c \\ 
		b = c
	\end{align*}
\end{proof}
