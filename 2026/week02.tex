\subsection{Week 2 (Jan 8--Jan 14)}

\dayentry{2026-01-08}{Rings}
\begin{theorem} 
	Let $f:R \to S$ be a homomorphism of rings, Then 
	\begin{align*}
		&\text{(1)}\quad f(0_R) = 0_S \\
		&\text{(2)}\quad f(-a) = -f(a) \quad \forall a \in R \\
		&\text{(3)}\quad f(a - b) = f(a) - f(b) \quad \forall a,b \in R
	\end{align*}
\end{theorem} 
\begin{proof} 
	(1) As $f$ is a homomorphism of rings, we have the following: 
	\begin{align*} 
		f(0_R) + f(0_R) &= f(0_R + 0_R) \\
		f(0_R) + f(0_R) &= f(0_R) \quad [0_R + 0_R = 0_R \,\text{in} \, R] \\
		f(0_R) + f(0_R) & = f(0_R) + 0_S \, [f(0_R) + 0_S = f(0_R)\, \text{in} \, S] \\
		f(0_R) &= 0_S
	\end{align*} 

	(2) Note that: $f(a) + f(-a) = f(a + (-a)) = f(0_R) = 0_S$ by (1) and $f$ is a homomorphism. So, $f(-a)$ is a solution to the equation $f(a) + x = 0_S$. But as $S$ is a ring, there exists only one unique solution to this equation which is $-f(a)$. Therefore by [theorem to be proved at a later date], $f(-a) = -f(a)$. 

	(3) As $f$ is a homomorphism and by (2), we have the following: 
	\begin{align*}
		f(a - b) &= f(a + (-b)) \\
		&= f(a) + f(-b) \\
		&= f(a) - f(b) 
	\end{align*} 
\end{proof}

\dayentry{2026-01-09}{Rings}
\begin{theorem} 
	Let $f: R \to S$ be a homomorphism of rings. If $R$ is a ring with identity and $f$ is surjective, then 
	\begin{align*}
		&\text{(1)}\quad S \, \text{is a ring with identity}\, f(1_R). \\
		&\text{(2)}\quad \text{Whenever }  u \, \text{is a unit in} \, R, \, \text{then} \, f(u) \, \text{is a unit in} \, S \, \text{and} \, f(u)^{-1} = f(u^{-1}).  
	\end{align*}
\end{theorem}
\begin{proof} 
	(1) Let $s$ be some element in $S$. As $f$ is surjective, there exists some $r \in R$ such that $f(r) = s$. Then we have that 
	\begin{align*}
		s * f(1_R) = f(r)f(1_R) = f(r * 1_R) = f(r) = s
	\end{align*}
	Similarly $f(1_R) * s = s$. Therefore $1_S = f(1_R)$.

	(2) As $u$ is a unit, there exists some $v \in R$ such that $uv = 1_R = vu$. By (1), we have that $f(u)f(v) = f(uv) = f(1_R) = 1_S$. Similarly, $vu = 1_R$ which implies that $f(v)f(u) = 1_S$. Therefore $f(u)$ is a unit in $S$ with inverse $f(v)$. Said differently $f(u)^{-1} = f(v)$. As $v$ is the inverse of $u$ ($v = u^{-1}$), we see that $f(u)^{-1} = f(v) = f(u^{-1})$. 
\end{proof} 

\dayentry{2026-01-10}{Groups}
\begin{theorem} 
	Every ring is an abelian group under addition. 
\end{theorem} 
\begin{proof} 
	The first five axioms for a ring are identical to the five axioms for an abelian group, with addition as the operation, the identity element being $0_R$, and the inverse for any element $a \in R$ being $-a$. 
\end{proof} 

\dayentry{2026-01-11}{Graphs} 
\begin{theorem} \label{prop:ClosedWalkCoherentCycle}
If $W = e_1,..., e_n$ is a closed walk in $D$, then $W$ contains a coherent cycle.
\end{theorem}
\begin{proof}
Consider the subgraph $H$ of edges $e_1,..., e_n$. Thus, no vertex is a source or sink in $H$. Now let $P$ be a longest path in $H$. Let $v$ denote the head of $P$. Because no vertex is a source or sink in $H$, there is $e=(v,w)$ for some $w$. Because $P$ has a maximum length $w \in P$, $P \cup (v,w)$ contains a coherent cycle. 
\end{proof}

\dayentry{2026-01-12}{Graphs} 
\begin{theorem} \label{prop:WalkYieldsPath}
If $u,v\in V(D)$ with $u\neq v$ and $W = e_1,..., e_n$ is a $uv$-walk. Then $W$ contains the edges of a $uv$-path. 
\end{theorem}
\begin{proof}
\textbf{Base Case:} Say $n=1$. Because $W$ is a $uv$-walk, its one edge is a $uv$-path. 

\noindent\textbf{Inductive Hypothesis:} Assume that for $W = e_1,..., e_n$, $W$ contains the edges of a $uv$-path. We need to show that there is a $uv$-path if $W = e_1,..., e_{n+1}$. Let $x_n$ be the head of $e_n$. If $v$ is contained within the $ux_n$-walk, then by the inductive hypothesis we are done. So, assume $v$ is not part of the $ux_n$-walk. By the inductive hypothesis, the $ux_n$-walk contains the edges of a $ux_n$-path. Adding $e_{n+1}$ to this $ux_n$-path makes a $uv$-path 
\end{proof}

\dayentry{2026-01-13}{Graphs} 
\begin{theorem} \label{thm:StrongConnectNoCut}
A directed graph D is strongly connected if and only if it has no coherent cut.  
\end{theorem}
\begin{proof}
($\Rightarrow$) 
Let $D$ be a directed graph that is strongly connected. By way of contradiction, assume there exists a coherent cut in D defined by $X \subseteq V(X)$, and with all edges in $\delta (X)$ having tails in $X$ without loss of generality. Because all edges in $\delta (D)$ point in the same direction, there would be no way to get from any vertex $ v \in V(D) - X$ to any vertex $x \in X$, so $D$ is not strongly connected, which is a contradiction. Thus, $D$ does not have a coherent cut. 
\bigbreak
\noindent 
($\Leftarrow$)
Let $D$ be a directed graph that is not strongly connected, so there exists two vertices $u, v \in V(D)$, such that there is no $uv$-path. Let $V_v \subseteq V(D)$ be the set of all vertices, $y$, for which there is a $yv$-path, and $V_u \subseteq V(D)$ be the set of all vertices, $x$ for which there is a $ux$-path. Let $M = V(D)-(V_v \cup V_u)$. Because there is no $uv$-path, $V_u \cap V_v = \emptyset$ by Theorem~\ref{prop:WalkYieldsPath}. Again by Theorem~\ref{prop:WalkYieldsPath}, all edges in $\delta (V_u)$ must going into $V_u$ by definition of $V_u, V_v$, and $M$. Thus, $\delta (V_u)$  is a coherent cut. 
\end{proof}

\dayentry{2026-01-14}{Graphs} 
\begin{theorem} \label{thm:CycleOrCut}
    Every $e \in D$ is a coherent cycle or a coherent cut. 
\end{theorem}
\begin{proof}
	Let $D$ be a directed graph. Let $e \in E(D)$. If e is in a coherent cycle, then we are done, so assume that $e$ is not in a coherent cycle, say $e=(u,v)$. Let $V_v \subseteq V(D)$ be the set of all vertices, $y$, for which there is a $vy$-path, $V_u \subseteq V(D)$ be the set of all vertices, $x$ for which there is a $xu$-path, and let $M = V(D)-(V_v \cup V_u)$. If there were a $vu$-walk, then by Theorem~\ref{prop:WalkYieldsPath}, there would be a coherent cycle containing $e$. So, it must be the case that there is no $vu$-walk, thus $V_u \cap V_v = \emptyset$. Again by Theorem~\ref{prop:WalkYieldsPath}, all edges in $\delta (V_u)$ must be going into $V_v \cup M$ by definition of $V_u, V_v$, and $M$, creating a coherent cut.
\end{proof}

\dayentry{2026-01-15}{Graphs} 
\begin{theorem} \label{thm:AcyclicPartialOrder}
If $D$ is acyclic, then $\le$ is a partial order set on $V(D)$. 
\end{theorem}
\begin{proof}
Let $D$ be a directed, acyclic graph. To show that $\le$ is a partial order set on $V(D)$, we must prove the following three properties.

\noindent\textbf{Reflexive}: For any $u \in V(D)$, there is a length $0$ coherent $uu$-path, so $u \le u$. 

\noindent\textbf{Anti-Symmetry}: Consider $u,v \in V(D)$. By way of contradiction, suppose $u \le v$ and $v \le u$, and $u \neq v$. So, there exists a $uv$-path, $P$, and a $vu$-path, $Q$, by the definition of $\le$. Now, $P \cup Q$ is a closed walk in $D$. So, there is a coherent cycle in $P \cup Q$ by Proposition 2, a contradiction. 

\noindent\textbf{Transitivity}: Consider $u, v, w \in V(D)$. Suppose $u \le v$ and $v \le w$. So there is a $vu$-path and a $wv$-path. Because you can get from $w$ to $v$ and from $v$ to $u$ there must be a $wu$-walk. So, there is a $uw$-path by Proposition 3. Thus, $u \le w$.
\noindent Thus, $\le$ is a partial order set set on $V(D)$. 
\end{proof}
